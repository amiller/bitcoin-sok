\section{An Introduction to Bitcoin}

Bitcoin's underlying mechanism.

Mining, transaction propagation.

Introduction of currency. Use of exchanges, market places. Financial ecosystem.

\subsection{A Explanation of Bitcoin in Steps}

\paragraph{Step 1: Imagine there’s a trusted third party (TTP)}
 Each participant is (somehow) assigned a single key pair which also acts as an address. The TTP allows anyone to send it messages and gives all participants a consistent view. Each participant (i.e., address) has some balance. We will defer the question of initial balances and where these balances come from.

If Alice (address $A$) wants to send Bob (address $B$) value $v$, she sends to the TTP a signed statement, called a transaction, saying “address $A$ transfers value $v$ to address $B$.” The TTP verifies each transaction (checks that $v ≤ balanceA$) and ignores the ones that don’t verify.

\paragraph{Step 2: Removing the TTP}

If Alice (address $A$) wants to send Bob (address $B$) value $v$, she broadcasts a signed statement to all other participants saying “address $A$ transfers value $v$ to address $B$.” Once every 10 minutes, the participants themselves (replacing the TTP) conduct a protocol to a) propose a batch of valid transactions, called a block, and b) vote on these proposals to select one.

Well known voting protocols have been studied, under some standard assumptions. First, it is assumed that a majority of the participants are honest. Second, it’s assumed that honest parties are able to broadcast synchronously to every other participant. Dishonest participants, on the other hand, are assumed to be able to equivocate, sending one message to some participants and a different message (or nothing) to others. Underlying all this is the fundamental assumption that there is a fixed set of participants that is known to everyone.

\footnote{For efficiency, there are randomized protocols.\anote{survey of randomized consensus, breaking n^2 barrier}}

\paragraph{Step 3: Removing limits on addresses}

In reality, participants don’t have preassigned identities, so the assumption that there is a fixed set of participants is untenable. Therefore we replace that assumption with a much weaker one: Imagine that there’s a magic token that lands on a random participant (chosen uniformly) at regular intervals. \footnote{The magic token is recognizable and unforgeable: if a participant receives a message from a participant holding the token, she knows that the sender did in fact hold the token.}

As before, participants broadcast each transaction to all other participants. When a participant gets the token she collects all the transactions she’s seen (in the most recent token interval), verifies them, and broadcasts the block.

\anote{Description of the hash chain data structure, how participants process blocks, longest chain rule, and how this leads to probabilistic consensus if the majority of the identities are honest.}

Note that participants can now create key pairs for themselves -- as many as they like. We have severed the link between participants and addresses. The only assumption is that the token selects participants uniformly at random, and hence selects an honest participant with a probability greater than $\frac{1}{2}$.

\paragraph{Step 4: Removing the token: proof-of-work}

Participants compete for possession of the token by racing to solve a proof-of-work based on hashing. Thus, participants acquire the token with a probability proportional to the fraction of Bitcoin’s hash power that they control. \footnote{there is a random nonce, so that each participant works on a different part of the problem space.} Attaching the proof-of-work to the block proves possession of the token, which is purely imaginary.

Now, instead of the majority of participants being honest, we require that the majority of hash power be controlled by honest participants. Except for that change, the argument for consensus in Step 3 holds here as well.

\paragraph{Step 5: Incentives}

To better justify the assumption that a majority of participants (by CPU power) are honest, we provide an incentive for honest behavior. This is done by adding value to the balance of an address chosen by each participant that finds a proof-of-work solution. \footnote{This is also the mechanism by which the values money are initially distributed, in an arguably fair way.} \footnote{Everything described so far is merely an abstract system for transferring and tracking balances. For the incentive system to work, these balances must be interpreted as money.}

This suggests a bootstrapping argument. If the transaction log is secure (has integrity and availability), then it will be useful as money, and therefore value added to an account makes a worthwhile incentive. This in turn will encourage a large amount of participation, which ensures the transaction log is secure.

This appears to be a virtuous cycle. On the other hand, neither system can exist without the other. If the integrity or the availability of the transaction log can be easily attacked, then we cannot expect it to maintain much value as a currency. Likewise, if the currency is not valuable as money, then it will not be an effective incentive for encouraging sufficient participation to prevent successful attacks.


\subsection{System Attack Model and Consensus Requirements}

The Bitcoin mining consensus proof from the whitepaper, assuming a majority of the hash power is ``honest''. Bitcoin's solution to the consensus problem is novel, although the proof known so far is for the wrong model (honest rather than altruistic). Given a set of rules, in a distributed system, the problem remains of how to choose a correct order of operations.

Although, so far, the economic incentive system seems to work, there are only heuristics and no clear model. As best we can tell, the reasoning is circular. Incentive system is needed to make the system secure. The built-in currency that rewards participants has to be valuable for the incentive to be meaningful. The currency is valuable only if the system is secure. The ``death spiral.''


\subsection{Bitcoin as an Abstract Platform}

What functionality does Bitcoin provide in the abstract? Originally presented as a mechanism for online ``payments,'' in the abstract it is a much more versatile platform.

The following is an abstract description of Bitcoin's goals, independent of system assumptions or implementation:
\begin{itemize}
\item (Append-only Log with Stabilizing-Consistency) The network eventually stabilizes to a single global view of a linear ordering of transactions. Unlike standard consistency, there is not necessarily a final time at which it's guaranteed to be settled. The more time passes, the more likely that a current view is settled. Given a model of attacker strength, and security probability, one can calculate an amount of time to wait before considering a prefix to be authoritative.
\item (Validation) Only valid transactions are committed to the log, where 'validity' is a function of the prior history of transactions.
\item (Fairness) After expending sufficient resources (i.e., by paying a fee), any transaction is eventually committed to the log.
\end{itemize}

Stabilizing consistency is weaker than a typical distributed system. In a typical distributed system, you receive an acknowledgment at some finite time that the transaction has been committed. In Bitcoin, there is no such acknowledgment, and users must user their own discretion about how long to wait to consider a transaction committed.

A typical definition of fairness or liveness would specify that *any* transaction should eventually accepted, without reference to adequate payment. However, since Bitcoin operates in a model with no established identities, this would be impossible due to the ability for an attacker to create sybil identities and flood the system.

The view of Bitcoin as a platform is that the currency is only inherently useful to the extent that it allows you to pay for transaction fees. In this sense it is comparable to postage stamps.

\subsection{Bitcoin's Model Assumptions}

Bitcoin's assumptions are weaker than in a standard distributed system - most notably, there are no pre-established identities. On the other hand, the assumptions are about the rational preferences of population - specifically that they respond to incentives, even when the incentives are inconclusive. Seems to be circular?


\subsection{How Changes are Applied}

Discussion of governance and the need for social out-of-band consensus to agree on rule changes. Soft forks, hard forks, and policy. Very few hard forks in history.

Discussion on the forums and mailing list that provide.

Ability for altcoins to develop.


\subsection{Towards an Economic Model}

Pooled mining and infrastructure investment.

A history of pooled mining, and what we can infer empirically about the motivations of miners? Prefer low variance.
